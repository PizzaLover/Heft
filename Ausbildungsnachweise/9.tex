\Titelzeile{9}{24.10}{28.10.2016}{1}
\Tabelle{
	\Inhalt{Montag}{
	\textbullet~ 1./2. St. BWL: Rechtsformen von Unternehmen, Warum braucht es Unternehmen?\par
	\textbullet~ 1./2. St. BWL: Lohnabgaben auf Seite des Arbeitgebers und Arbeitnehmers\par
	\textbullet~ 3./4. St. WiKo: Fachkräftemangel, Ausbildungsordnung, Fortbildungen \par
	\textbullet~ 5./6. St. SAE (AE): Assoziationen und Entitäten mit vielen Beispielen\par
	\textbullet~ 5./6. St. SAE (AE): Primärer und sekundärer Schlüssel, Testdatenbank in Access\par
	\textbullet~ 8./9. St. SAE (AE): SQL Befehle in Access (Create Table, Select, ...)\par
	\textbullet~ 
	}{}{O.P: 6,0}
	\Inhalt{Dienstag}{
	\textbullet~ 1./2. St. ITST: Umrechnung aus Dual-, Hexadezimal- und Dezimalsystem\par
	\textbullet~ 3./4. St. ITST: ASCII Code, letzte Übungen zu Zahlensystemumrechnung\par
	\textbullet~ 5./6. St. E-B: Formal/Informal conversations, phrases, present/past tense \par
	\textbullet~ \par
	\textbullet~ \par
	\textbullet~ \par
	\textbullet~ 
	}{}{O.P: 4,5}
	\Inhalt{Mittwoch}{
	\textbullet~ 1./2. St. ITST: Druckertreiber (PCL, PGL, PS), GDI, Farbgebung bei Laserdruckern \par
	\textbullet~ 3./4. St. ITST: Schutz vor Gefahren des elektrischen Stroms, rechtliche Lage\par
	\textbullet~ 5./6. St. BWL: Handelsregister, Kaufmann nach HGB, Firmengrundsätze\par
	\textbullet~ 5./6. St. BWL: Einzelunternehmungen (z.B. Vor- und Nachteile)\par
	\textbullet~ \par
	\textbullet~ \par
	\textbullet~ 
	}{}{O.P.: 4,5}
	\Inhalt{Donnerstag}{
	\textbullet~ 1./2. St. ITST: Elektrische Leistung, Elektrische Energie, Stromkosten\par
	\textbullet~ 3./4. St. ITST: Elektrische Grundschaltung mit Aufgaben\par
	\textbullet~ 16-18 Uhr: Interne Schulung (Verhalten beim Kunden, Bedarfsanalyse, Regeln)\par
	\textbullet~ \par
	\textbullet~ \par
	\textbullet~ \par
	\textbullet~ 
	}{}{O.P: 5,0}
	\Inhalt{Freitag}{
	\textbullet~ 1./2. St. Rel: Wie Erfahrungen im Hinblick auf ethische Grundhaltungen das - \par
	\textbullet~ 1./2. St. Rel: Leben eines Menschen verändern kann (Film)\par
	\textbullet~ 3./4. St. SAE (AE): JAVA: Programmierun Entfernung zweier Punkte (mit Eingabe)\par
	\textbullet~ 5./6. St. BWL: Merkmale und Eigenschaften der OHGs, KGs, GmbH und UG\par
	\textbullet~ 8./9. St. SAE (AE): HTML5 konformer Code, Codechecker, neue Website anfangen\par
	\textbullet~ \par
	\textbullet~ 
	}{}{O.P.: 6,0}
	\Inhalt{Samstag}{
	\textbullet~ \par
	\textbullet~ \par
	\textbullet~ \par
	\textbullet~ \par
	\textbullet~ \par
	\textbullet~ \par
	\textbullet~ 
	}{}{}
}{\Unten{}{
	\textbullet~ Berufsschule
}{0}{26}}
\Unterschrift
\newpage
