\Titelzeile{98}{02.07}{06.07.2018}{2}
\Tabelle{
	\Inhalt{Montag}{
	\textbullet~ 1./2. St. BWL: Klausur (Betriebsabrechnungsbogen, Zuschlagskalkulation)\par
	\textbullet~ 3./4. St. WiKo: Dispokredit, Formen, Kosten\par
	\textbullet~ 5./6. St. SAE: Datenbanken Wiederholungsblatt des Jahres anfertigen\par
	\textbullet~ 8./9. St. SAE (AE): UML Diagramme, Prüfungsaufgaben\par
	\textbullet~ \par
	\textbullet~ \par
	\textbullet~ 
	}{}{O.P.: 6,0}
	\Inhalt{Dienstag}{
	\textbullet~ 1./2. St. ITST: Klausur (VLANs, Routing, DHCP)\par
	\textbullet~ 3./4. St. ITST: VoIP, Einführung und Recherche\par
	\textbullet~ 5./6. St. GK: Klausurwiederholung (Geschichte Deutschlands, Verfassungen, DDR)\par
	\textbullet~ 8./9. St. E-B: Sprechen üben, mündliche Noten, Klausurbesprechung\par
	\textbullet~ \par
	\textbullet~ \par
	\textbullet~ 
	}{}{O.P.: 6,0}
	\Inhalt{Mittwoch}{
	\textbullet~ 1./2. St. ITST: Klausur (Schnittstellen, USB, parallel, seriell, Bussysteme)\par
	\textbullet~ 3./4. St. ITST: Film über die Entwicklung der Betriebssysteme\par
	\textbullet~ 5./6. St. BWL: Absatzrechnungen, Deckungsbeträge mit Excel berechnen\par
	\textbullet~ 8./9. St. GK: Klausur (Geschichte Deutschland, Verfassungen, DDR, Stasi)\par
	\textbullet~ \par
	\textbullet~ \par
	\textbullet~ 
	}{}{O.P.: 6,0}
	\Inhalt{Donnerstag}{
	\textbullet~ 1./2. St. ITST: Klausur (Extended und Standard ACLs, Spoofing)\par
	\textbullet~ 3./4. St. ITST: Gruppenarbeit für Präsentation zum Thema Denial of Service\par
	\textbullet~ 5./6. St. GK: Film zur Stasiüberwachung\par
	\textbullet~ \par
	\textbullet~ \par
	\textbullet~ \par
	\textbullet~ 
	}{}{O.P.: 4,5}
	\Inhalt{Freitag}{
	\textbullet~ 1./2. St. Rel: Arten von Gerichtigkeit, Bürgerkriege, Seltene Erden Industrie\par
	\textbullet~ 3./4. St. SAE (AE): UML Erzeugung und GUI Programmierung\par
	\textbullet~ 5./6. St. BWL: Absatzzuschlagskalkulation, Aufgaben mit Excel\par
	\textbullet~ \par
	\textbullet~ \par
	\textbullet~ \par
	\textbullet~ 
	}{}{O.P.: 4,5}
	\Inhalt{Samstag}{
	\textbullet~ \par
	\textbullet~ \par
	\textbullet~ \par
	\textbullet~ \par
	\textbullet~ \par
	\textbullet~ \par
	\textbullet~ 
	}{}{}
}{\Unten{}{
	\textbullet~ Berufsschule
}{0}{27}}
\Unterschrift
\newpage
