\Titelzeile{113}{15.10}{19.10.2018}{3}
\Tabelle{
	\Inhalt{Montag}{
	\textbullet~ 1./2. St. D: EU Gründungsstaaten, EU Verträge und Institutionen\par
	\textbullet~ 3./4. St. BWL: Präsentationen zu Marketingstrategien\par
	\textbullet~ 5./6. St. SAE: Geometrische Figuren mithilfe der Frame Klasse zeichnen\par
	\textbullet~ 8./9. St. SAE: Abschlussaufgabe in HTML bearbeiten und besprechen\par
	\textbullet~ \par
	\textbullet~ \par
	\textbullet~ 
	}{}{O.P.: 6,0}
	\Inhalt{Dienstag}{
	\textbullet~ 1./2. St. ITST: Abschlussaufgaben bearbeiten und Lösungen besprechen\par
	\textbullet~ 3./4. St. ITST: Weitere Abschlussaufgaben besprechen (Winter 2018)\par
	\textbullet~ 5./6. St. E-B: Freies Schreiben üben, Gruppenarbeit über Cyber Attacks\par
	\textbullet~ 8./9. St. E-B: Besprechung Cyber Attacks, Film über Cyberkriminalität\par
	\textbullet~ \par
	\textbullet~ \par
	\textbullet~ 
	}{}{O.P.: 6,0}
	\Inhalt{Mittwoch}{
	\textbullet~ 1./2. St. ITST: Datenschutz Fortführung\par
	\textbullet~ 3./4. St. ITST: Datenschutz und bestehende bzw. zukünftige Datensammlungen\par
	\textbullet~ 5./6. St. BWL: Abschlussaufgaben bearbeiten und besprechen\par
	\textbullet~ 8./9. St. SAE: Besprechung zu Linux Zertifikat, Einführung\par
	\textbullet~ \par
	\textbullet~ \par
	\textbullet~ 
	}{}{O.P.: 6,0}
	\Inhalt{Donnerstag}{
	\textbullet~ 1./2. St. ITST: Abschluss zu Routingprotokollen (OSPF, EIGRP)\par
	\textbullet~ 3./4. St. ITST: Einführung zu Spanning-Tree Protokoll, Erste Übung mit PT\par
	\textbullet~ 5./6. St. WiKo: Rechtliche Grundlagen der sozialen Marktwirtschaft erörtern\par
	\textbullet~ \par
	\textbullet~ \par
	\textbullet~ \par
	\textbullet~ 
	}{}{O.P.: 4,5}
	\Inhalt{Freitag}{
	\textbullet~ 1./2. St. WiKo: Grundlagen der Marktwirtschaft am Grundgesetz\par
	\textbullet~ 3./4. St. SAE: Struktogramme anhand einer Abschlussprüfung geübt\par
	\textbullet~ 5./6. St. Rel: Menschliche Entscheidungen vor der Ethik\par
	\textbullet~ 8./9. St. SAE: Vorbereitung für Klausur, Fragerunde\par
	\textbullet~ \par
	\textbullet~ \par
	\textbullet~ 
	}{}{O.P.: 6,0}
	\Inhalt{Samstag}{
	\textbullet~ \par
	\textbullet~ \par
	\textbullet~ \par
	\textbullet~ \par
	\textbullet~ \par
	\textbullet~ \par
	\textbullet~ 
	}{}{}
}{\Unten{}{
	\textbullet~ Berufsschule
}{0}{28,5}}
\Unterschrift
\newpage
