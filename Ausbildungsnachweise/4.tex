\Titelzeile{4}{19.09}{23.09.2016}{1}
\Tabelle{
	\Inhalt{Montag}{
	\textbullet~ 1./2. St. WiKo: Gegenseitigkeit der Berufsausbildung, Berufetrends\par
	\textbullet~ 3./4. St. WiKo: Lehrverträge, Pflichten und Kündigungsinformationen\par
	\textbullet~ 5./6. St. SAE: Datenbanken ->Einführung, Aufgaben, Datenbankwerkzeuge, Arten \par
	\textbullet~ 5./6. St. SAE: Datenbanksprache, Entity-Relation-Diagram mit diversen Beispielen\par
	\textbullet~ 8./9. St. SAE: Berufskolleg Vergleichswertetest (60 Min)\par
	\textbullet~ \par
	\textbullet~ 
	}{}{O.P.: 6,0}
	\Inhalt{Dienstag}{
	\textbullet~ 1./2. St. ITST: Übungsaufgaben zur Digitalisierung, Vorstellung Zahlensysteme\par
	\textbullet~ 1./2. St. ITST: Generelle Umwandlung von Zahlensysteme ins Dezimalsystem\par
	\textbullet~ 3./4. St. ITST: Dualsystem, Hexadezimalsystem, diverse Übungen\par
	\textbullet~ 5./6. St. E-B: Textverständnis, Paraphrasieren, Passiv- und Aktivsätze\par
	\textbullet~ 8./9. St. E-B: Hörverständnis (Bitcoin, Firmenstruktur)\par	
	\textbullet~ 8./9. St. E-B: Textproduktion über eigene Tätigkeiten in der Ausbildung, Vokabeln\par	
	\textbullet~ 
	}{}{O.P.: 6,0}
	\Inhalt{Mittwoch}{
	\textbullet~ 1./2. St. SAE: Weiterführung Druckerarten, Arten der Tintenstrahldrucker\par
	\textbullet~ 3./4. St. SAE: Funktionsprinzip Laserdrucker, Farbschemen, Vor- und Nachteile\par
	\textbullet~ 5./6. St. BWL: Betrieb und erwerbswirtschftliche/öffentliche Unternehmung\par
	\textbullet~ 5./6. St. BWL: Betriebsarten, vollkommene Märkte\par
	\textbullet~ 8./9. St. D: Sprechen und Zuhören, Zusammenfassen von Informationen\par
	\textbullet~ 8./9. St. D: Beispielfirma Aufbau und Einführung\par
	\textbullet~ 
	}{}{O.P.: 6,0}
	\Inhalt{Donnerstag}{
	\textbullet~ 1./2. St. ITST: Ohmsches Gesetz, Stromwirkungen, Zusammenhänge elektr. Größen\par
	\textbullet~ 3./4. St. ITST: Ohmsche Widerstände, Berechnungen (Spannung, Stromstärke, ...)\par
	\textbullet~ 5./6. St. GK: Rechte und Pflichten eines Auszubildenden früher und heute\par
	\textbullet~ 5./6. St. GK: Schlüsselqualifikationen, lebenslanges Lernen\par
	\textbullet~ \par
	\textbullet~ \par
	\textbullet~ 
	}{}{O.P.: 4,5}
	\Inhalt{Freitag}{
	\textbullet~ 1./2. St. Religion: Diskussion über moralisches Handeln (4 Beispielsituationen)\par
	\textbullet~ 3./4. St. SAE (AE): Java und Eclipse USB Konfiguration, HelloWorld\par
	\textbullet~ 3./4. St. SAE (AE): Datentypen und deren Unterschiede\par
	\textbullet~ 5./6. St. BWL: Die drei Wirtschaftssektoren und deren zeitliche Entwicklung\par
	\textbullet~ 8./9. St. SAE (AE): Einführung HTML, Test-Website mit Stylesheet nach Plan\par
	\textbullet~ \par
	\textbullet~ 
	}{}{O.P.: 6,0}
	\Inhalt{Samstag}{
	\textbullet~ \par
	\textbullet~ \par
	\textbullet~ \par
	\textbullet~ \par
	\textbullet~ \par
	\textbullet~ \par
	\textbullet~ 
	}{}{}
}{\Unten{}{
	\textbullet~ Berufsschule
}{0}{28.5}}
\Unterschrift
\newpage
