\Titelzeile{74}{15.01}{19.01.2018}{2}
\Tabelle{
	\Inhalt{Montag}{
	\textbullet~ 1./2. St. WiKo: Verbraucherschutzgesetze und Geld\par
	\textbullet~ 3./4. St. GK: Klausur (Gewaltenteilung, Wahlen, Bürgerinitiativen)\par
	\textbullet~ 5./6. St. SAE: Normalisierungsformen bei Datenbanken, Übungen mit Access\par
	\textbullet~ 8./9. St. SAE (AE): Klausurkorrektur, Übungen zu überschriebenen Methoden\par
	\textbullet~ \par
	\textbullet~ \par
	\textbullet~ 
	}{}{O.P.: 6,0}
	\Inhalt{Dienstag}{
	\textbullet~ 1./2. St. ITST: Klausur (Netzwerkdesign, Broadcast- /Kollisionsdomänen)\par
	\textbullet~ 3./4. St. ITST: Packet Tracer Übungen zu Routern und Switchen, Paketwege\par
	\textbullet~ 5./6. St. E-B: Textverstehen mit zwei Übungsaufgaben, Vokabeln\par
	\textbullet~ 8./9. St. E-B: Präpositionen, temporale und lokale Angaben (Ort, Zeit)\par
	\textbullet~ \par
	\textbullet~ \par
	\textbullet~ 
	}{}{O.P.: 6,0}
	\Inhalt{Mittwoch}{
	\textbullet~ 1./2. St. SAE: Klausur (Strukturierte Verkabelung, Vorteile, PoE, Protokolle)\par
	\textbullet~ 3./4. St. SAE (AE): Prüfungskriterien, Projekt Programmierung und Vorstellung\par
	\textbullet~ 5./6. St. BWL: Erfolgskonten, Aufwände und Erträge, Ergänzungen Buchführung\par
	\textbullet~ 8./9. St. D: Falsche Werbung, Irreführungen, Rechtliche Situation\par
	\textbullet~ \par
	\textbullet~ \par
	\textbullet~ 
	}{}{O.P.: 6,0}
	\Inhalt{Donnerstag}{
	\textbullet~ 1./2. St. ITST: Klausur (statische und dynamische Routingprotokolle, AD, Metrik)\par
	\textbullet~ 3./4. St. ITST: Übungen (Packet Tracer) zu OSPF, Eigenschaften besprechen\par
	\textbullet~ 5./6. St. GK: Prozedur der Gesetzgebung und die einzelnen Rollen der Organe\par
	\textbullet~ \par
	\textbullet~ \par
	\textbullet~ \par
	\textbullet~ 
	}{}{O.P.: 4,5}
	\Inhalt{Freitag}{
	\textbullet~ 1./2. St. Rel: Utopien in der heutigen Zeit, Risiken der Digitalisierung\par
	\textbullet~ 3./4. St. SAE (AE): Strings verschachteln, Objekthandling von Strings\par
	\textbullet~ 5./6. St. SAE: Website Fehler korrigieren, Formulare, Emailverknüpfung\par
	\textbullet~ \par
	\textbullet~ \par
	\textbullet~ \par
	\textbullet~ 
	}{}{O.P.: 4,5}
	\Inhalt{Samstag}{
	\textbullet~ \par
	\textbullet~ \par
	\textbullet~ \par
	\textbullet~ \par
	\textbullet~ \par
	\textbullet~ \par
	\textbullet~ 
	}{}{}
}{\Unten{}{
	\textbullet~ Berufsschule
}{0}{27}}
\Unterschrift
\newpage
