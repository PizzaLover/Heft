\Titelzeile{3}{12.09}{16.09.2016}{1}
\Tabelle{
	\Inhalt{Montag}{
	\textbullet~ 1./2. Stunde: Begrüßung, Allgemeine Informationen, Stundenplan, Blockplan\par
	\textbullet~ 3./4. Stunde: Hausordnung, (Ab-)Mahnungen, Nutzungsordnung, genereller Ablauf\par
	\textbullet~ 5./6. Stunde: Haus- und Raumführung \par
	\textbullet~ \par
	\textbullet~ \par
	\textbullet~ \par
	\textbullet~ 
	}{}{O. P.: 4.5}
	\Inhalt{Dienstag}{
	\textbullet~ 1./2. St. ITST: Cisco Zertifikatinformationen, Klausurenablauf, Bewertungssystem\par
	\textbullet~ 3./4. St. ITST: Informationsgrößen, Übertragungsverfahren (Analog/Digital)\par
	\textbullet~ 5./6. St. E-B: Kennenlernrunde, Bitcoin Diskussion mit gegebenem Text\par
	\textbullet~ \par
	\textbullet~ \par
	\textbullet~ \par
	\textbullet~ 
	}{}{O.P.: 4.5}
	\Inhalt{Mittwoch}{
	\textbullet~ 1./2. St. BWL/ITST: Druckerarten und Funktionen (Impact, Non Impact)\par
	\textbullet~ 3./4. St. SAE/ITST: Konrad Zuse Dokumentation\par
	\textbullet~ 5./6. St. BWL: Bedürfnispyramide, Marktverhalten (Diskussion) \par
	\textbullet~ 5./6. St. BWL: Ökonomisches Prinzip (Anfang)\par
	\textbullet~ \par
	\textbullet~ \par
	\textbullet~ 
	}{}{O.P.: 4.5}
	\Inhalt{Donnerstag}{
	\textbullet~ 1./2. St. ITST: Elektrische Grundgrößen, Ladungsfluss, Ladungsberechnung \par
	\textbullet~ 3./4. St. ITST: Elektrische Grundbegriffe, Schaltungen, Stromdichten, Spannungen\par
	\textbullet~ 5./6. St. GK: Erwartungen, Wünsche und Ängste beim Thema Ausbildung\par
	\textbullet~ 5./6. St. GK: Schlüsselqualifikationen, lebenslanges Lernen\par
	\textbullet~ \par
	\textbullet~ \par
	\textbullet~ 
	}{}{O.P.: 4.5}
	\Inhalt{Freitag}{
	\textbullet~ 1./2. St. Religion: Themenauswahl, ethisch-korrektes Handeln (Diskussion)\par
	\textbullet~ 3./4. St. SAE (AE): Einführung in Java (Grundaufbau, Klassen, Methoden, ...)\par
	\textbullet~ 5./6. St. BWL: Marktmodelle und Marktarten, Währungen und der Euro\par
	\textbullet~ 8./9. St. SAE (AE): Einführung HTML, Aufzählung Programmiersprachen\par
	\textbullet~ 8./9. St. SAE (AE): Interpreter- und Compilersprachen\par
	\textbullet~ \par
	\textbullet~ 
	}{}{O.P.: 6}
	\Inhalt{Samstag}{
	\textbullet~ \par
	\textbullet~ \par
	\textbullet~ \par
	\textbullet~ \par
	\textbullet~ \par
	\textbullet~ \par
	\textbullet~ 
	}{}{}
}{\Unten{}{
	\textbullet~
}{0}{24}}
\Unterschrift
\newpage
