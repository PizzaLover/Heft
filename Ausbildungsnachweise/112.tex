\Titelzeile{112}{08.10}{12.10.2018}{3}
\Tabelle{
	\Inhalt{Montag}{
	\textbullet~ 1./2. St. D: Besprechung Themen Abschlussprüfung, Textverständnis\par
	\textbullet~ 3./4. St. BWL: Marktformen und Marktstrategien\par
	\textbullet~ 5./6. St. SAE: Klasse Frame, draw-Methode, Koordinatensystem zeichnen\par
	\textbullet~ \par
	\textbullet~ \par
	\textbullet~ \par
	\textbullet~ 
	}{}{O.P.: 4,5}
	\Inhalt{Dienstag}{
	\textbullet~ 1./2. St. ITST: Stundeplan besprechen, Organisatorisches, Projektinfos\par
	\textbullet~ 3./4. St. ITST: Projektbesprechung, Bewertungskriterien\par
	\textbullet~ 5./6. St. D: Charakterisierung, Übungen zu Textverständnis\par
	\textbullet~ 8./9. St. E-B: Freies Schreiben üben und Sätze bilden\par
	\textbullet~ \par
	\textbullet~ \par
	\textbullet~ 
	}{}{O.P.: 6,0}
	\Inhalt{Mittwoch}{
	\textbullet~ 1./2. St. ITST: Datenschutzgrundverordnung (Datenschutz, Datensicherheit)\par
	\textbullet~ 3./4. St. ITST: Gefahren des Datenschutzes, rechtliche Entscheidungen\par
	\textbullet~ 5./6. St. BWL: Soziale Marktwirtschaft, Entwicklung seit dem 2. Weltkrieg\par
	\textbullet~ 8./9. St. SAE: Formen, Banküberweisung in HTML-Format erstellen\par
	\textbullet~ \par
	\textbullet~ \par
	\textbullet~ 
	}{}{O.P.: 6,0}
	\Inhalt{Donnerstag}{
	\textbullet~ 1./2. St. BWL: Produktpräsentation vorbereiten, Marketingstrategien\par
	\textbullet~ 3./4. St. WiKo: Weiterführende Entwicklung der sozialen Marktwirtschaft\par
	\textbullet~ 5./6. St. SOL: Ganzeinheitliche Aufgabe 1 als Übung bearbeiten\par
	\textbullet~ \par
	\textbullet~ \par
	\textbullet~ \par
	\textbullet~ 
	}{}{O.P.: 4,5}
	\Inhalt{Freitag}{
	\textbullet~ 1./2. St. GK: Gründung der EU, Vor- und Nachteile, Meinungen\par
	\textbullet~ 3./4. St. SAE: Gezeichnetes Koordinatensystem mit Schleifen beschriften\par
	\textbullet~ 5./6. St. Rel: Religionskritik in der heutigen Zeit, Entwicklung\par
	\textbullet~ 8./9. St. SAE: Rechnen und Besprechen von Abschlussaufgaben\par
	\textbullet~ \par
	\textbullet~ \par
	\textbullet~ 
	}{}{O.P.: 6,0}
	\Inhalt{Samstag}{
	\textbullet~ \par
	\textbullet~ \par
	\textbullet~ \par
	\textbullet~ \par
	\textbullet~ \par
	\textbullet~ \par
	\textbullet~ 
	}{}{}
}{\Unten{}{
	\textbullet~ Berufsschule
}{0}{27}}
\Unterschrift
\newpage
