\Titelzeile{57}{25.09}{29.09.2017}{2}
\Tabelle{
	\Inhalt{Montag}{
	\textbullet~ 1./2. St. WiKo: Vertragsrecht, Vertragsarten, Haftbarkeiten \par
	\textbullet~ 3./4. St. WiKo: Anwendungsbeispiele aus der Realität, Rechte von Minderjährigen\par
	\textbullet~ 5./6. St. SAE (AE): SQL-Abfragen, referentielle Integrität, ER-Diagramme\par
	\textbullet~ 8./9. St. SAE (AE): GGT, Konstruktor, Abfragen via Java.Util\par
	\textbullet~ \par
	\textbullet~ \par
	\textbullet~ 
	}{}{O.P.: 6,0}
	\Inhalt{Dienstag}{
	\textbullet~ 1./2. St. ITST: Geswitchte Netze, hierachisches Netzwerkdesgin\par
	\textbullet~ 3./4. St. ITST: Access-, Distribution-, Corelayer; Designprinzipien\par
	\textbullet~ 5./6. St. D: Schriftlich im Beruf kommunizieren, Anforderungen und Regeln\par
	\textbullet~ 8./9. St. E-B: Indirekte Rede mit verschiedenen Zeitformen, diverse Übungen\par
	\textbullet~ \par
	\textbullet~ \par
	\textbullet~ 
	}{}{O.P.: 6,0}
	\Inhalt{Mittwoch}{
	\textbullet~ 1./2. St. SAE: Vernetzung, zukünftige Entwicklung von Netzen, Ausfallsicherheit\par
	\textbullet~ 3./4. St. SAE: Koaxialverkabelung, Busstrukturen, Main-Frames, Entwicklung\par
	\textbullet~ 5./6. St. ITST: Organisatorisches, Merkmale von Funktionsgruppen\par
	\textbullet~ 8./9. St. D: Gruppenarbeit Protokollaufbau und Beispiel, Vortragen\par
	\textbullet~ \par
	\textbullet~ \par
	\textbullet~ 
	}{}{O.P.: 6,0}
	\Inhalt{Donnerstag}{
	\textbullet~ 1./2. St. ITST: Routerfunktionen, Routen, Subnetzkommunikation\par
	\textbullet~ 3./4. St. ITST: Packet Tracer: Statische Routen eintragen, Subnetting\par
	\textbullet~ 5./6. St. GK: Abgasskandal, politische Stellungnahmen der Parteien\par
	\textbullet~ \par
	\textbullet~ \par
	\textbullet~ \par
	\textbullet~ 
	}{}{O.P.: 4,5}
	\Inhalt{Freitag}{
	\textbullet~ 1./2. St. SAE (AE): HTML Scaling Boxes, Farbdesign, Validatoren\par
	\textbullet~ 3./4. St. SAE (AE): Diverse Algorithmen für Bruchrechnung mit Abfragen\par
	\textbullet~ \par
	\textbullet~ \par
	\textbullet~ \par
	\textbullet~ \par
	\textbullet~ 
	}{}{O.P.: 3,0}
	\Inhalt{Samstag}{
	\textbullet~ \par
	\textbullet~ \par
	\textbullet~ \par
	\textbullet~ \par
	\textbullet~ \par
	\textbullet~ \par
	\textbullet~ 
	}{}{}
}{\Unten{}{
	\textbullet~ Berufsschule
}{0}{25,5}}
\Unterschrift
\newpage
