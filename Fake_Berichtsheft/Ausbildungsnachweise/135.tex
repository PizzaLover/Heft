\Titelzeile{134}{18.03}{22.03.2019}{3}
\Tabelle{
	\Inhalt{Montag}{
	\textbullet~ 1./2. St. D: Testklausur zur Prüfungsvorbereitung\par
	\textbullet~ 3./4. St. BWL: Prüfungsaufgaben durchrechnen\par
	\textbullet~ 5./6. St. SAE: SAE Aufgaben aus GA 1 und 2 bearbeiten\par
	\textbullet~ \par
	\textbullet~ \par
	\textbullet~ \par
	\textbullet~ 
	}{}{O.P.: 4,5}
	\Inhalt{Dienstag}{
	\textbullet~ 1./2. St. ITST: Vorträge Gruppenarbeiten\par
	\textbullet~ 3./4. St. ITST: Eigener Vortrag zum Thema Virtualisierungen\par
	\textbullet~ 5./6. St. E-B: Gespräche führen, Abschluss des Unterrichts \par
	\textbullet~ 8./9. St. GK: Besprechung von Themen zur Prüfungsvorbereitung\par
	\textbullet~ \par
	\textbullet~ \par
	\textbullet~ 
	}{}{O.P.: 6,0}
	\Inhalt{Mittwoch}{
	\textbullet~ 1./2. St. ITST: Wiederholung, Prüfungsaufgaben\par
	\textbullet~ 3./4. St. ITST: Wiederholung, Prüfungsaufgaben\par
	\textbullet~ 5./6. St. BWL: Besprechung WiKo-Klausur, Übungen\par
	\textbullet~ 8./9. St. SAE: Linux Essentials Chapter weiterbearbeiten\par
	\textbullet~ \par
	\textbullet~ \par
	\textbullet~ 
	}{}{O.P.: 6,0}
	\Inhalt{Donnerstag}{
	\textbullet~ 1./2. St. ITST: Wiederholung USV, Berechnungen\par
	\textbullet~ 3./4. St. ITST: Subnetting, STP-Protokoll\par
	\textbullet~ 5./6. St. WiKo: Prüfungsaufgaben durchführen und besprechen\par
	\textbullet~ 8./9. St. BWL: Zeit für Prüfungsaufgaben\par
	\textbullet~ \par
	\textbullet~ \par
	\textbullet~ 
	}{}{O.P.: 6,0}
	\Inhalt{Freitag}{
	\textbullet~ 1./2. St. BWL: Abschließende Übungen durchführen, Erhalt weiterer Aufgaben\par
	\textbullet~ 3./4. St. SAE: Datenbanken Übungen durchführen\par
	\textbullet~ 5./6. St. Rel: Note Klausur erhalten, Film zur künstlichen Intelligenz\par
	\textbullet~ 8./9. St. SAE: Java Programm Bierschaumanalyse\par
	\textbullet~ \par
	\textbullet~ \par
	\textbullet~ 
	}{}{O.P.: 6,0}
	\Inhalt{Samstag}{
	\textbullet~ \par
	\textbullet~ \par
	\textbullet~ \par
	\textbullet~ \par
	\textbullet~ \par
	\textbullet~ \par
	\textbullet~ 
	}{}{}
}{\Unten{}{
	\textbullet~ Berufsschule
}{0}{28,5}}
\Unterschrift
\newpage
