\Titelzeile{139}{15.04}{19.04.2019}{3}
\Tabelle{
	\Inhalt{Montag}{
	\textbullet~ WS: 37 Postfächer und Accounts angelegt, Kontakte, Verteiler\par
	\textbullet~ WS: ,Postfach-Berechtigungen übernommen, AD Benutzer ohne Postfach erstellt\par
	\textbullet~ WS: Übernahme Postfachinhalte vor Ort planen, Skript erstellt\par
	\textbullet~ VOS: An beiden Standorten HDDs angeschlossen, um Daten vorab zu kopieren\par
	\textbullet~ \par
	\textbullet~ \par
	\textbullet~ 
	}{}{8}
	\Inhalt{Dienstag}{
	\textbullet~ WS: Alle Ordner und Freigaben anlegen, Berechtigungsgruppen definieren\par
	\textbullet~ WS: Benutzer Berechtigungsgruppen hinzufügen\par
	\textbullet~ WS: Daten mit Einrichtungsleiter aussortieren, Ordnerstrukturen anlegen\par
	\textbullet~ VOS: Abholung der Platten\par
	\textbullet~ WS: Daten in die Ordnerstrukturen kopiert, Plan für den Abgleich vor Ort erstellt\par
	\textbullet~ \par
	\textbullet~ 
	}{}{8}
	\Inhalt{Mittwoch}{
	\textbullet~ WS: Mailarchivierung installiert, aktiviert und konfiguriert\par
	\textbullet~ WS: Notwendige Daten dokumentiert, Archivierungstest -> OK\par
	\textbullet~ WS: Auf allen Servern Monitoring eingerichtet und getestet\par
	\textbullet~ WS: Neue Dokumentation vervollständigt und abschließen\par
	\textbullet~ WS: Gerät für Auslieferung zwecks Postfachexport konfigurieren\par
	\textbullet~ WS: Vorbereitende Maßnahmen durch Softwarebetreuer\par
	\textbullet~ 
	}{}{8}
	\Inhalt{Donnerstag}{
	\textbullet~ WS: Vorbereitung zusätzliches Netz Firewall für Parallelbetrieb der Server\par
	\textbullet~ WS: Planung der Arbeiten bei Auslieferung (Ablaufplan erstellen)\par
	\textbullet~ WS: Vollständigkeit aller Arbeiten sicherstellen\par
	\textbullet~ WS: Systeme zusammenpacken und für die Mitnahme vorbereiten\par
	\textbullet~ WS: Zusätzliche Materialien richten\par
	\textbullet~ \par
	\textbullet~ 
	}{}{8}
	\Inhalt{Freitag}{
	\textbullet~ Feiertag\par
	\textbullet~ \par
	\textbullet~ \par
	\textbullet~ \par
	\textbullet~ \par
	\textbullet~ \par
	\textbullet~ 
	}{}{0}
	\Inhalt{Samstag}{
	\textbullet~ \par
	\textbullet~ \par
	\textbullet~ \par
	\textbullet~ \par
	\textbullet~ \par
	\textbullet~ \par
	\textbullet~ 
	}{}{}
}{\Unten{}{
	\textbullet~ Ausbildungsbetrieb
}{0}{32}}
\Unterschrift
\newpage
