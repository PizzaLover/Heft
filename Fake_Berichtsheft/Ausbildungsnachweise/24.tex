\Titelzeile{24}{06.02}{10.02.2017}{1}
\Tabelle{
	\Inhalt{Montag}{
	\textbullet~ 3./4. St. ITST: Klausurvorbereitung Digitaltechnik, CCNA Besprechung\par
	\textbullet~ 5./6. St. SAE (AE): SQL-Abfragen (Simultanabfrage, Joins), Klausurvorbereitung\par
	\textbullet~ 8./9. St. SAE (AE): While-Schleifen, Übungsaufgaben (Teiler, Zufallszahlen)\par
	\textbullet~ \par
	\textbullet~ \par
	\textbullet~ \par
	\textbullet~ 
	}{}{O.P.: 4,5}
	\Inhalt{Dienstag}{
	\textbullet~ 1./2. St. ITST: Klausur Digitaltechnik (Umrechnen Zahlensysteme, Logikgatter...)\par
	\textbullet~ 3./4. St. ITST: CCNA Anmeldebesprechung, Einführung Netzwerktechnik\par
	\textbullet~ 6./7. St. D: Klausurvorbereitung, Gruppenarbeit, Telefongespräche, Ohrenmodell\par
	\textbullet~ \par
	\textbullet~ \par
	\textbullet~ \par
	\textbullet~ 
	}{}{O.P.: 4,5}
	\Inhalt{Mittwoch}{
	\textbullet~ 1./2. St. SAE: Prozessketten, Prozessoptimierungen, Prozessdiagramme\par
	\textbullet~ 3./4. St. SAE: Zeitmanagement, Kalendereinträge, Wochenplanung\par
	\textbullet~ 5./6. St. ITST: Netzwerkkomponenten und ihre Aufgaben, Kabel vs Wireless\par
	\textbullet~ 8./9. St. D: Klausur über Gruppenarbeiten, 4-Ohren-Modell, Charakterisierung\par
	\textbullet~ \par
	\textbullet~ \par
	\textbullet~ 
	}{}{O.P.: 6,0}
	\Inhalt{Donnerstag}{
	\textbullet~ 1./2. St. ITST: Klausurbesprechung Elektrotechnik, Cisco Paket Tracer\par
	\textbullet~ 3./4. St. ITST: Netzwerksimulation, Pings versenden, Gateways und DNS eintragen\par
	\textbullet~ 5./6. St. GK: Familie und Familienpolitik, rechtliche Situation, Probleme\par
	\textbullet~ \par
	\textbullet~ \par
	\textbullet~ \par
	\textbullet~ 
	}{}{O.P.: 4,5}
	\Inhalt{Freitag}{
	\textbullet~ 3./4. St. SAE (AE): While-Schleife Fortführung, tabellarische Textausgabe\par
	\textbullet~ 5./6. St. Rel: Religionsklausur (Todesstrafe, ethische Grundhaltungen)\par
	\textbullet~ 8./9. St. SAE (AE): HTML Eingabefelder, Datensubmitting, Formen\par
	\textbullet~ \par
	\textbullet~ \par
	\textbullet~ \par
	\textbullet~ 
	}{}{O.P.: 4,5}
	\Inhalt{Samstag}{
	\textbullet~ \par
	\textbullet~ \par
	\textbullet~ \par
	\textbullet~ \par
	\textbullet~ \par
	\textbullet~ \par
	\textbullet~ 
	}{}{}
}{\Unten{}{
	\textbullet~ Berufsschule
}{0}{24}}
\Unterschrift
\newpage
