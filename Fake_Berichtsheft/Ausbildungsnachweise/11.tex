\Titelzeile{11}{07.11}{11.11.2016}{1}
\Tabelle{
	\Inhalt{Montag}{
	\textbullet~ 1./2. St. WiKo: Arbeitsunfälle in Deutschland, Jugendarbeitsschutzgesetz \par
	\textbullet~ 1./2. St. WiKo: Personengerechter Arbeitsschutz, Erkrankungen, Schilderrätsel\par
	\textbullet~ 3./4. St. WiKo: Unterschiede technischer und sozialer Arbeitsschutz\par
	\textbullet~ 5./6. St. SAE (AE): Vorteile, Nachteile, Darstellungen des Relationenmodells\par
	\textbullet~ 5./6. St. SAE (AE): SQL-Abfragen mit SELECT, FROM, WHERE\par
	\textbullet~ 8./9. St. SAE (AE): Javaprogramme zu Primzahlen, Summen, ... + Hausarbeit\par
	\textbullet~ 
	}{}{O.P.: 6,0}
	\Inhalt{Dienstag}{
	\textbullet~ 1./2. St. ITST: Arten von Codes, Zahlencodes, BCD-Code\par
	\textbullet~ 1./2. St. ITST: TK-Leitungen, Logische Grundverknüpfungen: Nicht, Und\par
	\textbullet~ 3./4. St. ITST: -> Oder, Nicht-Und, Nicht-Oder, Antivalenz, Äquivalenz\par
	\textbullet~ 5./6. St. E-B: "Coversations with 3 people", Hörverstehen zu Computerteilen \par
	\textbullet~ 8./9. St. E-B: "4 important presidents of the us", "Debate Trump and Clinton"\par
	\textbullet~ 8./9. St. E-B: Computerbauanleitung auf Englisch übersetzen\par
	\textbullet~ 
	}{}{O.P.: 6,0}
	\Inhalt{Mittwoch}{
	\textbullet~ 1./2. St. ITST: Unfallvermeidung, Direktes und indirektes berühren, Erste Hilfe\par
	\textbullet~ 3./4. St. ITST: Begriffsbestimmungen, Elektromagnetische Verträglichkeit (EMV)\par
	\textbullet~ 5./6. St. BWL: Vergleich Deutschland/USA: Thema Politik - Präsidentschaftswahl\par
	\textbullet~ \par
	\textbullet~ \par
	\textbullet~ \par
	\textbullet~ 
	}{}{O.P.: 4,5}
	\Inhalt{Donnerstag}{
	\textbullet~ 1./2. St. ITST: Reihenschaltung, Formelherleitungen, diverse Aufgaben\par
	\textbullet~ 3./4. St. ITST: Parallelschaltung, Multisimaufbau, diverse Aufgaben, Hausarbeit\par
	\textbullet~ 5./6. St. GK: Familie und Familienpolitik (Grundlagen, Rechtliches, ...)\par
	\textbullet~ \par
	\textbullet~ \par
	\textbullet~ \par
	\textbullet~ 
	}{}{O.P.: 4,5}
	\Inhalt{Freitag}{
	\textbullet~ 1./2. St. Rel: Regeln der Gesellschaft, Rechtfertigung Todesstrafe + Informationen \par
	\textbullet~ 3./4. St. SAE (AE): If- und While-Anweisung, hus-Struktogramme + Formalitäten\par
	\textbullet~ 3./4. St. SAE (AE): Relationale Operatoren, Programmierung eines Portorechners\par
	\textbullet~ 5./6. St. BWL: Aktiengesellschaft, Aktienkurse, Zielgruppen, Rechte des Aktionärs\par
	\textbullet~ 8./9. St. SAE (AE): HTML Tabellenprogrammierung mit .css Datei, Interaktivität\par
	\textbullet~ \par
	\textbullet~ 
	}{}{O.P.: 6,0}
	\Inhalt{Samstag}{
	\textbullet~ \par
	\textbullet~ \par
	\textbullet~ \par
	\textbullet~ \par
	\textbullet~ \par
	\textbullet~ \par
	\textbullet~ 
	}{}{}
}{\Unten{}{
	\textbullet~ Berufsschule
}{0}{27}}
\Unterschrift
\newpage
