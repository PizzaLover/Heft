\Titelzeile{118}{26.11}{30.11.2018}{3}
\Tabelle{
	\Inhalt{Montag}{
	\textbullet~ 1./2. St. D: Charakterisierungen, Dialoge, Briefe\par
	\textbullet~ 3./4. St. BWL: Zuschlagskalkulationen, Abschlussprüfungen\par
	\textbullet~ 5./6. St. SAE: Struktogramme, Rechnungen in Java, Korrekturen\par
	\textbullet~ \par
	\textbullet~ \par
	\textbullet~ \par
	\textbullet~ 
	}{}{O.P.: 4,5}
	\Inhalt{Dienstag}{
	\textbullet~ 1./2. St. ITST: Formalitäten, Besprechung Abschlussprüfung\par
	\textbullet~ 3./4. St. ITST: Redundanz und Link Aggregation \par
	\textbullet~ 5./6. St. D: Klausurvorbereitung (literarische Texte, Interpretation)\par
	\textbullet~ 8./9. St. E-B: Freitexte schreiben, Confusing Verbs\par
	\textbullet~ \par
	\textbullet~ \par
	\textbullet~ 
	}{}{O.P.: 6,0}
	\Inhalt{Mittwoch}{
	\textbullet~ 1./2. St. ITST: Erneut Datenschutz und Datensicherheit\par
	\textbullet~ 3./4. St. ITST: Sicherheitslücken aktueller Smart Home Geräte\par
	\textbullet~ 5./6. St. BWL: Kumulativer Umsatz, ABC-Analyse\par
	\textbullet~ 8./9. St. SAE: Linux Zertifikatsübungen durchführen\par
	\textbullet~ \par
	\textbullet~ \par
	\textbullet~ 
	}{}{O.P.: 6,0}
	\Inhalt{Donnerstag}{
	\textbullet~ 1./2. St. ITST: STP Algorithmus Einführung\par
	\textbullet~ 3./4. St. ITST: STP Algorithmus Funktion und Übungen\par
	\textbullet~ 5./6. St. BWL: Kostenvergleiche in Excel umsetzen\par
	\textbullet~ 8./9. St. WiKo: Märkte, gewollte und ungewollte Märkte\par
	\textbullet~ \par
	\textbullet~ \par
	\textbullet~ 
	}{}{O.P.: 6,0}
	\Inhalt{Freitag}{
	\textbullet~ 1./2. St. BWL: Besprechung SOL-Aufgaben, Handelsreisender und -vertreter\par
	\textbullet~ 3./4. St. SAE: Vergleich von Strings, Umsetzen von Aufgaben\par
	\textbullet~ 5./6. St. Rel: Existenz von Gott anhand Bibelstellen, Beistand\par
	\textbullet~ 8./9. St. SAE: Projektaufgabe (Aufgabe mit Java und SQL)\par
	\textbullet~ \par
	\textbullet~ \par
	\textbullet~ 
	}{}{O.P.: 6,0}
	\Inhalt{Samstag}{
	\textbullet~ 8-13 Uhr: Cisco Kurs, praktische Übungen\par
	\textbullet~ \par
	\textbullet~ \par
	\textbullet~ \par
	\textbullet~ \par
	\textbullet~ \par
	\textbullet~ 
	}{}{O.P.: 5,0}
}{\Unten{}{
	\textbullet~ Berufsschule
}{0}{33,5}}
\Unterschrift
\newpage
