\Titelzeile{103}{06.08}{10.08.2018}{2}
\Tabelle{
	\Inhalt{Montag}{
	\textbullet~ WS: Bearbeitung E-Mails, Besprechung mit der Disposition\par
	\textbullet~ FW: Lexware Server Probleme, Firewalleinstellungen anpassen\par
	\textbullet~ FW: Skripte zur Einbindung in Backupsoftware erstellen, da Fehler auftreten\par
	\textbullet~ VOS: Verkabelungsarbeiten, TK-Arbeiten (Telefone, Umleitungen)\par
	\textbullet~ VOS: Störungsanalyse Verkabelung, Durchklingeln der Leitungen\par
	\textbullet~ WS: Konvertierung XP-System in virtuellen Datenträger (.vhdx)\par
	\textbullet~ FW: Server läuft instabil -> Analyse: Defektes Backuplaufwerk
	}{}{9}
	\Inhalt{Dienstag}{
	\textbullet~ VOS: RDP-Verbindung konfigurieren, Anmeldeskripte und Shares angepasst\par
	\textbullet~ FW: Windows Integration Services installiert und geprüft -> OK\par
	\textbullet~ FW: Kaspersky Keys über KSC deployen und Verteilung prüfen\par
	\textbullet~ WS: Beratung eines Kunden bzgl. des bestehenden Backupkonzepts\par
	\textbullet~ FW: Anpassungen an Acronis Backup Software, Anbindung an Exchange\par
	\textbullet~ WS: Auf Produktionsrechner Treiberinstallationen durchgeführt\par
	\textbullet~ FW: Zwei neue Benutzer im AD inklusive neuer Mailadressen angelegt
	}{}{9}
	\Inhalt{Mittwoch}{
	\textbullet~ FW: Programmprobleme bei Kunde behoben -> OK\par
	\textbullet~ FW: Konfigurationen an Backup Agent durchgeführt -> OK\par
	\textbullet~ WS: Aufbereitung einer Kundendokumentation inkl. Übergabe\par
	\textbullet~ FW: Installation und Einrichtung Client-Software für Zeiterfassung\par
	\textbullet~ FW: Virtualbox Fehler analysieren und beheben\par
	\textbullet~ FW: SMTP-Daten Kunden bereitstellen, RAID Manager installieren\par
	\textbullet~ 
	}{}{9}
	\Inhalt{Donnerstag}{
	\textbullet~ FW: Anpassen White-List in KSE, kleine Anpassungen an der Konfiguration\par
	\textbullet~ VOS: Programmberechtigungen anpassen, Weiterleitungen setzen, Gerätestatus\par
	\textbullet~ VOS: Reservergerät vorbereiten und übergeben, Mailproblem iPad beheben\par
	\textbullet~ VOS: Globale Einstellungen für Outlook Phishing Add-in setzen \par
	\textbullet~ \par
	\textbullet~ \par
	\textbullet~ 
	}{}{9}
	\Inhalt{Freitag}{
	\textbullet~ FW: Anbindung Branchensoftware Mailversand an In-House Exchange\par
	\textbullet~ FW: Probleme mit Netzwerkkamera für Produktionsmaschine beheben\par
	\textbullet~ VOS: Umstellung Firma auf Glasfaser Anbindung, VPN, Forwardings, DynDNS\par
	\textbullet~ \par
	\textbullet~ \par
	\textbullet~ \par
	\textbullet~ 
	}{}{9,25}
	\Inhalt{Samstag}{
	\textbullet~ \par
	\textbullet~ \par
	\textbullet~ \par
	\textbullet~ \par
	\textbullet~ \par
	\textbullet~ \par
	\textbullet~ 
	}{}{}
}{\Unten{}{
	\textbullet~ Ausbildungsbetrieb
}{0}{45,25}}
\Unterschrift
\newpage